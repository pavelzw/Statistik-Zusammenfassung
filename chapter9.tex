\appendix
\section{Konvergenzarten}

\begin{karte}{Konvergenz von Zufallsvariablen}
Seien \(X, X_1, X_2, \ldots\) Zufallsvariablen mit Verteilungsfunktionen 
\(F_X, F_{X_1}, \ldots\).
\begin{itemize}
    \item Die Folge \(\set{X_n}\) konvergiert fast sicher gegen \(X\), falls 
    \[ P\left( \set{ \omega \in \Omega : \limes{n} X_n(\omega) = X(\omega) }\right) = 1. \]
    \item Die Folge \(\set{X_n}\) konvergiert stochastisch gegen \(X\), falls 
    \[ P\left( \abs{X_n - X} \geq \varepsilon \right) = 0 \;\forall \varepsilon > 0. \]
    \item Die Folge \(\set{X_n}\) konvergiert in Verteilung gegen \(X\), falls 
    \[ \limes{n} F_{X_n}(x) = F_X(x) \forall x\in C(F_x), \]
    wobei \(C(F_X)\) die Menge der Stetigkeitsstellen von \(F_X\) bezeichnet.
\end{itemize}
\end{karte}

\begin{karte}{Konvergenzarten Zusammenhänge und Rechenregeln}
Es gilt 
\[ X_n \oversett{f.s.}{\longrightarrow} X \Rightarrow X_n \overset{P}{\longrightarrow} X \Rightarrow X_n \overset{\mathcal{D}}{\longrightarrow} X. \]

Es sei \(c\in\R\). Dann gilt: 
\[ X_n \overset{\mathcal{D}}{\longrightarrow} x \Leftrightarrow X_n \overset{P}{\longrightarrow} c. \]

Es sei \(\abb{g}{\R}{\R}\) stetig. Dann gilt: 
\begin{align*}
    X_n \oversett{f.s.}{\longrightarrow} X \Rightarrow g(X_n) \oversett{f.s.}{\longrightarrow} g(X), \\
    X_n \overset{P}{\longrightarrow} X \Rightarrow g(X_n) \overset{P}{\longrightarrow} g(X), \\
    X_n \overset{\mathcal{D}}{\longrightarrow} X \Rightarrow g(X_n) \overset{\mathcal{D}}{\longrightarrow} g(X).
\end{align*}
\end{karte}

\begin{karte}{Konvergenz von Zufallsvektoren}
Seien \(Z, Z_1, \ldots\) \(k\)-dimensionale Zufallsvektoren. Die stochastische 
und fast sichere Konvergenz wird in diesem Fall komponentenweise definiert. 
Bei der Verteilungskonvergenz werden die Verteilungsfunktionen durch die entsprechenden 
multivariaten Verteilungsfunktionen ersetzt, also nicht komponentenweise!

Aus \(X_n \overset{\mathcal{D}}{\longrightarrow} X, Y_n \overset{\mathcal{D}}{\longrightarrow} Y\) folgt i. A. nicht 
\(X_n + Y_n \overset{\mathcal{D}}{\longrightarrow} X + Y \). Hier ist die Forderung \((X_n, Y_n)^T \overset{\mathcal{D}}{\longrightarrow} (X,Y)^T\)
nötig, um darauf schließen zu können.
\end{karte}

\begin{karte}{Lemma von Slutzky}
Für reelle Zufallsvariablen \(X, X_1, \ldots\) und \(Y_1, \ldots\) gelte \(X_n \overset{\mathcal{D}}{\longrightarrow} X\) und \(Y_n \overset{P}{\longrightarrow} c, c\in \R\). 
Dann gilt: 
\begin{enumerate}
    \item \(X_n + Y_n \overset{\mathcal{D}}{\longrightarrow} X+c\).
    \item \(Y_n \cdot X_n \overset{\mathcal{D}}{\longrightarrow} c\cdot X\).
    \item \(X_n / Y_n \overset{\mathcal{D}}{\longrightarrow} X/c\), falls \(c\neq 0\).
\end{enumerate}
\end{karte}