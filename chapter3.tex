\section{Testen}

\subsection{\(z\)-Test}

\begin{karte}{Modellierung}
\(x_1, \ldots, x_n\) Realisierungen von ZVen \(X_1, \ldots, X_n\). 
\(X_1, \ldots, X_n\) seien unabhängig. 
Annahme: \(X_i \sim \mathcal{N}(\mu, \sigma^2)\). Zu testen ist, ob diese Annahme begründet ist. 

Fragestellung: 
Ist \(\mu < \mu_0\) oder nicht?
Man möchte zeigen, dass \(\mu < \mu_0\) gilt. 

Aufstellen von Hypothesen: 
\(H_0\): \(\mu = \mu_0\) (oder \(\mu \geq \mu_0\))
\(H_1\): \(\mu < \mu_0\) (möchte man zeigen!)

Vorgehen: Berechne \(\bar{x}_n\). Entscheide aufgrund dessen ob die Hypothese \(H_0\) verworfen werden kann.
\end{karte}

\begin{karte}{Fehler 1. und 2. Art}
\begin{tabular}{p{15mm}||p{40mm}|p{40mm}}
& Test \(\mu = \mu_0\) \newline (\(H_0\) gilt) & Test \(\mu < \mu_0\) \newline (\(H_1\) gilt) \\
\hline
\(\mu = \mu_0\) \newline (\(H_0\) gilt) & richtige Entscheidung \newline \(1-\alpha=0.95\) & Fehler 1. Art \newline \(\alpha=0.05\) \\
\(\mu < \mu_0\) \newline (\(H_1\) gilt) & Fehler 2. Art \newline \(\beta=?\) & richtige Entscheidung \newline \(1-\beta = ?\)
\end{tabular}
\end{karte}

\begin{karte}{Testentscheidung}
Test entscheidet für \(H_1\): Sehr unwahrscheinlich, wenn in Realität \(H_0\) gilt. Folgerung: 
\(H_1\) wird in Realität gelten. \gqq{Man lehnt \(H_0\) auf dem Niveau \(\alpha\) ab!}

Test entscheidet für \(H_0\): Da über \(\beta\) nichts bekannt ist, ist keine weitere Aussage möglich. 
\gqq{Man lehnt \(H_0\) auf dem Niveau \(\alpha\) nicht ab!} oder \gqq{Die Daten stehen nicht im Widerspruch zu \(H_0\).}
\end{karte}

\begin{karte}{\(z\)-Test}
Unter \(H_0\) gilt: 
\[ \bar{X}_n \sim \mathcal{N}\left(\mu_0, \frac{\sigma^2}{n}\right) \Rightarrow T = \frac{\sqrt{n}(\bar{X}_n - \mu_0)}{\sigma} \sim \mathcal{N}(0,1). \]
Sei \(z_\alpha := \Phi^{-1}(\alpha)\).

\begin{itemize}
    \item Falls \(T \leq z_{0.05}\): \(H_0\) ablehnen.
    \item Falls \(T > z_{0.05}\): \(H_0\) nicht ablehnen.
\end{itemize}
Wegen \(P_{H_0}(T \leq z_{0.05}) = 0.05\) liegt ein Test zum \(5\%\)-Niveau vor.

Falls \(X_i\) nicht \(\mathcal{N}(\mu, \sigma^2)\)-verteilt ist, gilt nach dem ZGWS zumindest \(T \approx \mathcal{N}(0,1)\).
Der Test hält in diesem Fall asymptotisch das Niveau \(\alpha\) ein.
\end{karte}

\begin{karte}{Grundbegriffe}
Sei das statistische Modell \((\mathfrak{X}, (P_\vartheta)_{\vartheta\in \Theta})\) 
und die Zerlegung \(\Theta = \Theta_0 \cup \Theta_1\) mit \(\Theta_0 \cap \Theta_1 = \emptyset\).

Ein statistischer Test soll anhand der Daten eine Entscheidung zwischen \(\Theta_0\) und \(\Theta_1\) treffen.
\begin{itemize}
    \item Nullhypothese: \(H_0: \vartheta \in \Theta_0\)
    \item Alternativhypothese: \(H_1: \vartheta \in \Theta_1\)
    \item Ein Test zur Prüfung von \(H_0\) gegen \(H_1\) ist gegeben durch einen kritischen Bereich \(\mathcal{K} \subset \mathfrak{X}\) mit der Vorschrift: 
    \begin{align*}
        x \in \mathcal{K} &\Rightarrow \text{Entscheidung: } H_1 \\
        x \in \mathfrak{X}\setminus \mathcal{K} &\Rightarrow \text{Entscheidung: } H_0
    \end{align*}
\end{itemize}
\end{karte}

\begin{karte}{Umfang}
Die Funktion 
\[ g: \Theta \rightarrow [0,1], \vartheta \mapsto g(\vartheta) = P_\vartheta(X \in \mathcal{K}) = P_\vartheta(\text{Entscheidung für }H_1) \]
heißt Gütefunktion des Tests mit kritischem Bereich \(\mathcal{K}\).
Ideal wäre \(g(\vartheta) = \mathds{1}_{\Theta_1}\).

\(\sup_{\vartheta \in \Theta_0} g(\vartheta)\) heißt \textit{Umfang} des Tests.
Niveau \(\alpha\)-Tests werden so konstruiert, dass der Umfang möglichst nahe bei \(\alpha\) ist. 
Dadurch wächst die Wahrscheinlichkeit, dass der Test das Vorliegen der Alternative erkennt.
\end{karte}

\subsection{1-SP-\(t\)-Test}

\begin{karte}{Zweiseitiger Ein-Stichproben-\(t\)-Test}
Seien \(X_1, \ldots, X_n \oversett{uiv}{\sim} \mathcal{N}(\mu, \sigma^2)\), \(\mu, \sigma^2\) unbekannt.

\(H_0\): \(\mu = \mu_0\) gegen \(H_1\): \(\mu \neq \mu_0\).

Testgröße: 
\[ T(x_1, \ldots, x_n) = \frac{\sqrt{n} (\bar{x} - \mu_0)}{s} \]
Unter \(H_0\) gilt: 
\[ T(X_1, \ldots, X_n) \sim t_{n-1}. \]
\begin{align*}
    H_0 \text{ verwerfen,} & \text{falls } \abs{T} \geq t_{n-1;1-\frac{\alpha}{2}},
    \text{kein Widerspruch zu } H_0 & \text{falls } \abs{T} < t_{n-1;1-\frac{\alpha}{2}}.
\end{align*}
Es ist \begin{align*}
    \Theta &= \set{\vartheta = (\mu, \sigma^2): \mu \in \R, \sigma^2 > 0}, \\
    \Theta_0 &= \set{\vartheta \in \Theta: \mu = \mu_0}, \\
    \Theta_1 &= \set{\vartheta \in \Theta: \mu \neq \mu_0}.
\end{align*}
\end{karte}

\begin{karte}{Einseitiger Ein-Stichproben-\(t\)-Test}
    Seien \(X_1, \ldots, X_n \oversett{uiv}{\sim} \mathcal{N}(\mu, \sigma^2)\), \(\mu, \sigma^2\) unbekannt.
    
    \(H_0\): \(\mu = \mu_0\) (bzw. \(\mu \leq \mu_0\)) gegen \(H_1\): \(\mu > \mu_0\).
    
    Testgröße: 
    \[ T(x_1, \ldots, x_n) = \frac{\sqrt{n} (\bar{x} - \mu_0)}{s} \]
    Unter \(H_0\) gilt: 
    \[ T(X_1, \ldots, X_n) \sim t_{n-1}. \]
    \begin{align*}
        H_0 \text{ verwerfen,} & \text{falls } T \geq t_{n-1;1-\alpha},
        \text{kein Widerspruch zu } H_0 & \text{falls } T < t_{n-1;1-\alpha}.
    \end{align*}
    
    Alternative \( \mu < \mu_0 \):
    \(H_0\): \(\mu = \mu_0\) (bzw. \(\mu \geq \mu_0\)) gegen \(H_1\): \(\mu < \mu_0\).
    \begin{align*}
        H_0 \text{ verwerfen,} & \text{falls } T \leq t_{n-1;\alpha} = -t_{n-1;1-\alpha},
        \text{kein Widerspruch zu } H_0 & \text{falls } T > t_{n-1;\alpha} = -t_{n-1;1-\alpha}.
    \end{align*}
\end{karte}

\begin{karte}{Tests und Konfidenzintervalle}
Sei \( [\bar{X} - \frac{S}{\sqrt{n}} \cdot t_{n-1;1-\frac{\alpha}{2}}, \bar{X} + \frac{S}{\sqrt{n}} \cdot t_{n-1;1-\frac{\alpha}{2}}] \)
das \(1-\alpha\)-Konfidenzintervall für \(\mu\) mit Normalverteilungsannahme. 
Es gilt also \(P_\vartheta(I(X) \ni \mu) = 1-\alpha\). 

Zu testen ist 
\(H_0: \mu=\mu_0\) gegen \(H_1: \mu \neq \mu_0\).
Lehne \(H_0\) ab, falls \(i(X) \not\ni \mu_0\).

Aus einem Test lässt sich ebenso ein Konfidenzintervall konstruieren, falls für jedes \(\mu_0 \in \R\) getestet wird.
\end{karte}

\begin{karte}{Ein-Stichproben-Varianz-Test}
Sei \(\chi^2 := \frac{(n-1)S^2}{\sigma_0^2}\). 
In jedem der Fälle gilt \(\chi^2 \sim \chi_{n-1}^2 \) unter \(H_0\).
\begin{enumerate}
    \item \(H_0: \sigma^2 = \sigma_0^2\) gegen \(H_1: \sigma^2 > \sigma_0^2\). Lehne \(H_0\) ab, falls \(\chi^2 \geq \chi_{n-1;1-\alpha}^2\).
    \item \(H_0: \sigma^2 = \sigma_0^2\) gegen \(H_1: \sigma^2 < \sigma_0^2\). Lehne \(H_0\) ab, falls \(\chi^2 \leq \chi_{n-1;1-\alpha}^2\).
    \item \(H_0: \sigma^2 = \sigma_0^2\) gegen \(H_1: \sigma^2 \neq \sigma_0^2\). Lehne \(H_0\) ab, falls \(\chi^2 \leq \chi_{n-1;1-\frac{\alpha}{2}}^2\).
\end{enumerate}
\end{karte}

\subsection{Der \(p\)-Wert}

\begin{karte}{\(p\)-Wert}
Der \(p\)-Wert ist die Wahrscheinlichkeit, bei Gültigkeit der Hypothese etwas mindestens so Extremes zu beobachten wie das tatsächlich Beobachtete.

Für kritischen Bereich der Form \(T \geq c\) gilt 
\[ p^* = P_{H_0}(T \geq T(x)). \]

\(c\) wird so gewählt, dass \(P_{H_0}(T\geq c) = \alpha\).
Gilt nun \(p^* \leq \alpha\) folgt, dass \(T(x) \geq c \) gelten muss, also wird \(H_0\) auf dem Niveau \(\alpha\) verworfen. 
Gilt dagegen \(P^* > \alpha\), so muss \(T(x) < x\) gelten und \(H_0\) wird nicht verworfen.

\(p^*\) ist die kleinste Zahl, die man als Signifikanzniveau wählen kann, sodass der Test gerade noch zur Ablehnung von \(H_0\) führt.
\end{karte}

\subsection{Optimalität}

\begin{karte}{Bester Test}
Unter einem \textit{besten Test} für \(H_0: \vartheta \in \Theta_0\) gegen \(H_1: \vartheta \in \Theta_1\) 
versteht man einen Niveau-\(\alpha\)-Test, dessen Güte \(P_\vartheta(H_0 \text{ wird verworfen})\) für jedes \(\vartheta \in \Theta_1\) 
maximal ist.

Wir betrachten nur die Situation einer einfachen Hypothese \(H_0: \vartheta = \vartheta_0\) gegen 
eine einfache Alternative \(H_1: \vartheta = \vartheta_1\).
\end{karte}

\begin{karte}{Neyman-Pearson-Lemma}
Seien \(X_1, \ldots, X_n\) u. i. v. mit Dichte \(f_0\) unter \(H_0\) bzw. \(f_1\) unter \(H_1\). 
Die gemeinsame Dichte ist dann 
\[ h_k(x) = \prod_{i=1}^n f_k(x_i), \quad k \in \set{0,1}. \]

Der Neyman-Pearson-Test mit Testentscheid
\begin{align*}
    H_0 \text{ verwerfen, falls} & h_1(x) \geq c \cdot h_0(x), 
    H_0 \text{ nicht verwerfen, falls} & h_1(x) < c \cdot h_0(x).
\end{align*}
ist für ein \(c \in (0,\infty)\) der beste Test für \(H_0\) gegen \(H_1\) zum Niveau \\
\(\alpha = P_0(\text{NP-Test verwirft } H_0)\).
\end{karte}

\begin{karte}{Neyman-Pearson-Lemma 2}
Das NP-Lemma besagt im Wesentlichen, dass ein optimaler Test die Test-Statistik 
\[ T(x) = \frac{h_1(x)}{h_0(x)} \]
verwendet und \(H_0\) für \(T(x) \geq c\) verwirft, wobei \(x\) so zu wählen ist, dass 
die Wahrscheinlichkeit des Fehlers 1. Art gleich \(\alpha\) ist. 

Allerdings besagt der Satz nicht, dass für jedes \(\alpha \in (0,1)\) so ein \(c\) 
auch tatsächlich existiert.

Dies ist zumindest dann der Fall, wenn \(T(X)\) eine stetige Verteilung besitzt.
\end{karte}

\subsection{Likelihood-Quotienten-Tests}

\begin{karte}{Verallgemeinerte Likelihood-Quotienten-Testgröße}
Die \textit{(verallgemeinerte) Likelihood-Quotienten-Testgröße} wird definiert als 
\[ \Lambda(x) := \frac{ \sup_{\vartheta \in \Theta} h(x;\vartheta) }{ \sup_{\vartheta \in \Theta_0} h(x;\vartheta) } \]
Der \textit{(verallgemeinerte) Likelihood-Quotienten-Test} für 
\[ H_0: \vartheta \in \Theta_0 \text{ gegen } H_1: \vartheta \in \Theta \setminus \Theta_0 \]
verwirft \(H_0\) für große Werte von \(\Lambda\).
\end{karte}

\begin{karte}{Verteilungskonvergenz Schätzfolge}
Seien \(X_1, \ldots\) eine Folge von u. i. v. ZVen mit Dichte \(f(x, \vartheta_0), \vartheta_0 \in \Theta\) so, 
dass (R1)-(R5) erfüllt sind. Für die Fisher-Information gelte \(0 < I(\vartheta_0) < \infty\).
Für eine konsistente Schätzfolge \((\hat{\vartheta}_n)\) gelte 
\[ \sqrt{n}(\hat{\vartheta}_n - \vartheta_0) \overset{\mathcal{D}_{\vartheta_0}}{\longrightarrow} \mathcal{N}\left( 0, \frac{1}{I(\vartheta_0)} \right). \]
Dann gilt für die Folge der LQ-Statistiken \( (\Lambda_n(X))_{n\in\N} \)
\[ 2 \log(\Lambda_n) \overset{\mathcal{D}_{\vartheta_0}}{\longrightarrow} \chi_1^2. \]
\end{karte}

\begin{karte}{Verteilungskonvergenz LQ-Statistik}
Sind \(X_1, \ldots\) u. i. v. ZVen mit Dichte \(f(x;\vartheta)\), so gilt für die LQ-Statistik 
\[ \Lambda_n = \prod_{i=1}^n \frac{f(X_i; \hat{\vartheta}_n)}{f(X_i, \hat{\vartheta}_n^0)} \]
folgendes Resultat:
\[ 2 \log \Lambda_n \overset{\mathcal{D}}{\longrightarrow} \chi_r^2 \text{ unter } H_0, \]
wobei \(\hat{\vartheta}_n\) der ML-Schätzer ohne Restriktionen ist.
\end{karte}