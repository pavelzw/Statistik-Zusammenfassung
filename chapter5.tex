\section{Lineare Regression}

\subsection{Regressionsanalyse}

\begin{karte}{Lineares Regressionsmodell}
Regressionsanalyse modelliert den Zusammenhang zwischen 
\(y\), der Zielgröße, und \(x_1, \ldots, x_k\), den Regressoren. 
\(k=1\) ist einfache Regression, \(k>1\) ist multiple Regression.

Das \textit{lineare Regressionsmodell (LRM)} ist gegeben durch 
\[ y = \beta_0 + \beta_1 x_1 + \cdots + \beta_k x_k + \varepsilon. \]
Dabei sind \(\beta_0, \ldots, \beta_k\) unbekannte Parameter (\(\beta_0\) ist Intercept)
und \(\varepsilon\) ist ein additiver Fehler.

Dieses Modell soll an die Daten \(y_i, x_{i1}, \ldots, x_{ik}\) angepasst werden. 

\(x_{ij}\) seien feste Werte, nicht zufällig und \(\varepsilon_i\) ein Zufallsfehler mit 
\(E \varepsilon_i = 0, V(\varepsilon_i) = \sigma^2\) und \(C(\varepsilon_i, \varepsilon_j) = 0, i\neq j\).

Somit gilt 
\(EY_i = \beta_0 + \beta_1 x_{i1} + \cdots + \beta_k x_{ik}, V(Y_i) = \sigma^2, C(Y_i,Y_j) = 0\).
\end{karte}

\begin{karte}{Lineares Regressionsmodell Matrixschreibweise}
\[ Y = \begin{pmatrix} 
    Y_1 \\ \vdots \\ Y_n 
\end{pmatrix}, X = \begin{pmatrix}
    1 & x_{11} & \cdots & x_{1k} \\
    \vdots & \vdots & & \vdots \\
    1 & x_{n1} & \cdots & x_{nk} 
\end{pmatrix} \in \R^{n \times (k+1)}, 
    \varepsilon = \begin{pmatrix}
        \varepsilon_1 \\ \vdots \\ \varepsilon_n
\end{pmatrix}, \beta = \begin{pmatrix}
        \beta_0 \\ \vdots \\ \beta_k
\end{pmatrix}. \]
\[ Y = X \beta + \varepsilon. \]
mit \(E \varepsilon = 0\) und \(C(\varepsilon) = \sigma^2 I_n\).
\end{karte}

\subsection{Kleinste-Quadrate}

\begin{karte}{Orthogonalprojektion}
Sei \(\mathcal{L}(X)\) der von den Spalten von \(X\) aufgespannte Unterraum des \(\R^n\).
\(\hat{y}\) minimiert den Abstand von \(y\) zu \(\mathcal{L}(X)\), d. h. es gilt 
\[ || y - \hat{y} ||^2 = \min_{z \in \mathcal{L}(X)} || y - z||^2 = \min_{\beta \in \R^{k+1}} ||y-X\beta||^2. \]

Sei \(y\in \R^n\), \(V \) ein linearer Unterraum von \(\R^n\). \(\hat{y}\) heißt \textit{Orthogonalprojektion} von 
\(y\) auf \(V\), falls 
\[ \hat{y} \in V \text{ und } y - \hat{y} \in V^\bot. \]
\begin{enumerate}
    \item Es gilt \(||y - \hat{y}||^2 = \min_{z\in V} ||y-z||^2\).
    \item Die Orthogonalprojektion \(\hat{y}\) ist eindeutig bestimmt.
    \item Bilden die \(p\) Spalten der Matrix \(X\) eine Basis von \(V\) und existiert 
    somit \((X^T X)^{-1}\), so ist 
    \[ \hat{y} = Hy \text{ mit } H := X (X^T X)^{-1} X^T \in \R^{n\times n} \]
    die Orthogonalprojektion von \(y\) auf \(V\).
\end{enumerate}
\end{karte}

\begin{karte}{Kleinste-Quadrate-Schätzer}
Ist in einem LRM die Matrix \(X^T X\) invertierbar, so ist wegen \(\hat{y} = Hy = X (X^T X)^{-1} X^T y = X\hat{\beta}\) 
der Kleinste-Quadrate-Schätzer \(\hat{\beta}\) von \(\beta\) gegeben durch 
\[ \hat{\beta} = (X^T X)^{-1} X^T y. \]
\end{karte}

\begin{karte}{Prädiktionswerte, Residuen, RSS}
Die Komponenten \(y_i\) von \(\hat{y}\) heißen auch \textit{Prädiktionswerte}. 

\(\hat{\varepsilon} = y - \hat{y}\) heißt \textit{Residuenvektor}. 
\(||\hat{\varepsilon}||^2\) heißt \textit{Residuen-Quadratsumme}. Es gilt 
\[ ||\hat{\varepsilon}||^2 = y^T (I_n - H)y \]
wobei \(H\) die \gqq{Hut-Matrix} ist: \(H = X (X^T X)^{-1} X^T\). Es gilt \(HX = X\).
\end{karte}

\begin{karte}{Normalengleichung}
Aus einer analytischen Herleitung von \(\hat{\beta}\) (\(||y - X\beta||^2\) minimieren)
folgt die Normalengleichung 
\[ X^T X \beta = X^T y. \]
Für die praktische Berechnung lösen wir die Normalengleichung iterativ.
\end{karte}

\begin{karte}{Eigenschaften von \(\hat{\beta}\) und \(\hat{Y}\)}
\[E\hat{\beta} = \beta, C(\hat{\beta}) = \sigma^2 (X^T X)^{-1}, \]
\[ E \hat{Y} = X \beta, C(\hat{Y}) = \sigma^2. \]
\end{karte}

\begin{karte}{\(\hat{\sigma}^2\)}
\(\hat{\sigma}^2 = \frac{1}{n-p} ||\hat{\varepsilon}||^2 = \frac{1}{n-p} ||Y-\hat{Y}||^2\).
\(\hat{\sigma}^2\) ist erwartungstreu, d. h. \(E(\hat{\sigma}^2) = \sigma^2\).
\end{karte}

\begin{karte}{Standardabweichung von \(\hat{\beta}_j\)}
\[ \mathrm{se}(\hat{\beta}_j) := \hat{\sigma} \sqrt{ (X^T X)^{-1}_{j+1,j+1} }. \]
\end{karte}

\begin{karte}{Das Bestimmtheitsmaß \(R^2\)}
In einem LRM definiert man das \textit{Bestimmtheitsmaß} durch 
\[ R^2 := 1 - \frac{ \sum_{i=1}^n (y_i - \hat{y}_i)^2 }{ \sum_{i=1}^n (y_i - \bar{y}_n)^2 } = 1 - \frac{\mathrm{RSS}}{\mathrm{TSS}}. \]
Es gilt 
\[ R^2 = \frac{ \sum_{i=1}^n (\hat{y}_i - \bar{y}_i)^2 }{ \sum_{i=1}^n (y_i - \bar{y}_n)^2 } \text{ und somit } 0 \leq R^2 \leq 1. \]

\(\bar{y}\) ist die beste Vorhersage ohne Kenntnis von \(x\), \(\mathrm{TSS}\) der Vorhersagefehler ohne Kenntnis von \(x\). 
\(\hat{y}\) ist die beste Vorhersage mit Kenntnis von \(x\), \(\mathrm{RSS}\) der Vorhersagefehler mit Kenntnis von \(x\). 
\end{karte}

\begin{karte}{Empirischer Korrelationskoeffizient}
Der empirische Korrelationskoeffizient ist definiert durch 
\[ r_{xy} = \frac{\frac{1}{n-1} \sum (x_i - \bar{x})(y_i - \bar{y})}{s_x s_y}, \]
wobei \(s_x\) und \(s_y\) die Stichproben-Standardabweichung der \(x\)- bzw. \(y\)-Werte bezeichnet. 
Im einfachen LRM gilt \(R^2 = r_{xy}^2\).
\end{karte}

\begin{karte}{\(R^2\) im LRM ohne Inercept}
Die Definition von \(R^2\) ist nur sinnvoll, wenn es einen Intercept im Modell gibt. 

Wenn ein LRM ohne Intercept vorliegt, muss \(\bar{y}\) im Nenner weggelassen werden.

Das \textit{adjungierte Bestimmtheitsmaß} ist durch 
\[ R_a^2 := 1 - \frac{n-1}{n-p} (1-R^2) = 1 - \frac{\mathrm{RSS}/(n-p)}{\mathrm{TSS}/(n-1)} = 1 - \frac{\hat{\sigma}^2}{s_y^2}. \]
gegeben.
\end{karte}

\begin{karte}{Gauß-Markov-Theorem}
Sei \(a\in \R^p\). Dann ist \(T := a^T \hat{\beta}\) \textit{bester linearer erwartungstreuer Schätzer} 
(best linear unbiased estimtor, BLUE) für \(a^T \beta\).
\end{karte}

\subsection{Das klassische LM}

\begin{karte}{Annahmen}
Beim klassischen linearen Modell wird zusätzlich angenommen, dass die Fehler normalverteilt sind. 
Dann ist 
\[ \varepsilon_1, \ldots, \varepsilon_n \oversett{uiv}{\sim} \mathcal{N}(0, \sigma^2) \text{ bzw. } \varepsilon \sim \mathcal{N}_n(0, \sigma^2 I_n). \]

Im klassischen linearen Modell gilt 
\[ Y = X \beta + \varepsilon \sim \mathcal{N}_n(X \beta, \sigma^2 I_n). \]
\[ \hat{\beta} = (X^T X)^{-1} X^T Y \sim \mathcal{N}_p(\beta, \sigma^2 (X^T X)^{-1}). \]
\[ \hat{Y} \sim \mathcal{N}_n(X\beta, \sigma^2 H). \]
Für \(\hat{\varepsilon}\) gilt \(\hat{\varepsilon} \sim \mathcal{N}_n(0, \sigma^2(I_n - H))\).
\end{karte}

\begin{karte}{Eigenschaften im klassischen LRM}
Sei \(\hat{\sigma}_{ML}^2 := \frac{\hat{\varepsilon}^T \hat{\varepsilon}}{n}\). Im klassischen LRM gilt: 
\begin{enumerate}
    \item \((\hat{\beta}, \hat{\sigma}_{ML}^2)\) ist Maximum-Likelihood-Schätzer für \((\beta, \sigma^2)\).
    \item Für die Likelihood-Funktion an der Stelle \((\hat{\beta}, \hat{\sigma}_{ML}^2)\) gilt: 
    \[ L_y(\hat{\beta}, \hat{\sigma}_{ML}^2) = (2\pi e \hat{\sigma}_{ML}^2)^{-n/2}. \]
    \item \(\hat{\beta}\) und \(\hat{\sigma}_{ML}^2\) sind stochastisch unabhängig.
    \item Es gilt 
    \[ \frac{n \hat{\sigma}_{ML}^2}{\sigma^2} = \frac{(n-p)\hat{\sigma}^2}{\sigma^2} = \frac{\hat{\varepsilon}^T \hat{\varepsilon}}{\sigma^2} \sim \chi_{n-p}^2. \]
\end{enumerate}
\end{karte}

\subsection{Tests im klassischen LM}

\begin{karte}{Test im klassischen linearen Modell}
Oft sind viele Regressoren vorhanden, sind alle nötig?

Obermodell: \(Y_i = \beta_0 + \beta_1 x_{i1} + \cdots + \beta_{p-1} x_{i p-1} + \varepsilon_i\)
Teilmodell: \(Y_i = \beta_0 + \beta_1 x_{i1} + \cdots + \beta_{r-1} x_{i r-1} + \varepsilon_i\), \(1 \leq r < p\).

Zu testen ist die Hypothese 
\[ H_0: \beta_r = \beta_{r+1} = \cdots = \beta_{p-1} = 0 \] 
gegen die Alternative \(H_1: \) nicht alle \(\beta_j\) sind gleich \(0\).

Falls \(H_0\) nicht verworden wird, können die entsprechenden Regressoren \(x_r, \ldots, x_{p-1}\) weggelassen werden.

Definiere \(X_r, \hat{\beta}_r, \hat{Y}_r, \hat{\varepsilon}_r, \hat{\sigma}_r^2, \mathrm{RSS}_r\) wie gewohnt. 

Da \(d := \hat{Y} - \hat{Y}_r\) und \(\hat{\varepsilon}\) senkrecht zueinander stehen, gilt 
\[ d^T d + \hat{\varepsilon}^T \hat{\varepsilon} = \hat{\varepsilon}_r^T \hat{\varepsilon}_r \text{ bzw. } d^T d = \mathrm{RSS}_r - \mathrm{RSS}. \]
\end{karte}

\begin{karte}{Likelihood-Quotienten-Statistik zum Vergleich von Teil- und Obermodell}
Es gilt 
\[ \Lambda_n = \frac{\max_{(\beta, \sigma^2) \in \Theta} L_y(\beta, \sigma^2)}{\max_{(\beta, \sigma^2) \in \Theta} L_y(\beta, \sigma^2)} 
= \left(  \frac{\mathrm{RSS}_r}{\mathrm{RSS}}  \right)^{n/2}. \]
\(H_0\) wird für große Werte von \(\Lambda_n\) abgelehnt.
\end{karte}

\begin{karte}{\(F\)-Test zum Vergleich von Teil- und Obermodell}
Unter \(H_0: \beta_r = \ldots = \beta_{p-1} = 0\) ist die Teststatistik 
\[ F = \frac{(\mathrm{RSS}_r - \mathrm{RSS}) / (p-r)}{ \mathrm{RSS} / (n-p) } \]
\(F\)-verteilt mit \(p-r\) Zähler- und \(n-p\) Nenner-Freiheitsgeraden.

Somit ergibt sich folgender Testentscheid zum Niveau \(\alpha\):
\begin{align*}
    H_0 \text{ verwerfen, falls } & F \geq F_{p-r, n-p; 1-\alpha}, \\
    H_0 \text{ nicht verwerfen, falls } & F < F_{p-r, n-p; 1-\alpha}.
\end{align*}
\end{karte}

\begin{karte}{Ein-Stichproben-\(t\)-Test zum Vergleich von Teil- und Obermodell}
Gehört der \(j\)-te Regressor ins Modell?

Es gilt mit \(b_{jj} := (X^T X)^{-1}_{j+1,j+1}\) 
\[ \hat{\beta}_j \sim \mathcal{N}(\beta_j, \sigma^2 b_{jj}) \Rightarrow \frac{\hat{\beta}_j - \beta_j}{\sigma \sqrt{b_{jj}}} \sim \mathcal{N}(0,1). \]

Weiter gilt \(\frac{\hat{\varepsilon}^T \hat{\varepsilon}}{\sigma^2} \sim \chi_{n-p}^2\).

Daraus folgt mit \(\mathrm{se}(\hat{\beta}_j) = \hat{\sigma} \sqrt{b_{jj}}\)
\[ \frac{\hat{\beta}_j - \beta_j}{ \mathrm{se}(\hat{\beta}_j)} \sim t_{n-p}. \]

Somit kann man \(H_0: \beta_j = 0\) auch mit der Testgröße 
\[ T := \frac{\hat{\beta}_j}{\mathrm{se}(\hat{\beta}_j)} \overset{H_0}{\sim} t_{n-p} \]
testen (mit zweiseitigem Ablehnbereich).

Es lässt sich zeigen, dass \(T^2 = F\) gilt.
\end{karte}

\begin{karte}{ANOVA-Tafel}
Wenn wir alle Regressoren testen, gibt es die sog. ANOVA-Tafel.

\begin{center}
\begin{tabular}{c|ccccc}
    & Freiheitsgerade & Quadratsumme & & Teststatistik & \(p\)-Wert \\
    \hline 
    Regression & \(p-1\) & \(\mathrm{TSS} - \mathrm{RSS}\) & \(\frac{\mathrm{TSS} - \mathrm{RSS}}{p-1}\) & \(F\) & \(p^*\) \\
    Residuen & \(n-p\) & \(\mathrm{RSS}\) & \(\frac{\mathrm{RSS}}{n-p}\) \\
    \hline 
    Gesamt & \(n-1\) & \(\mathrm{TSS}\)
\end{tabular}
\end{center}
\end{karte}

\subsection{Konfidenzbereiche für \(\beta\)}

\begin{karte}{Quaderförmige Konfidenzbereiche (\gqq{Bonferroni-Methode})}
Sei \(I_1\) ein \((1-\frac{\alpha}{2})\)-Konfidenzintervall für \(\beta_1\), 
also \(P(I_1 \ni \beta_1) = 1 - \frac{\alpha}{2}\).
Genauso \(I_2\). 
Wegen \(P(A_1 \cap A_2) \geq 1 - P(A_1^C) + P(A_2^C)\) ist 
\(I_1 \times I_2\) ein Konfidenzbereich für \((\beta_1, \beta_2)\) zur Wahrscheinlichkeit \(1-\alpha\).

Für den allgemeinen Fall mit \(d \geq 2\) wähle Analog \(I_j\) so, dass \(P(I_j \ni \beta_j) \geq 1 - \frac{\alpha}{d}\). 
So kommen wir auf einen quaderförmigen Konfidenzbereich \(I_1 \times \cdots \times I_d\) zur Wahrscheinlichkeit \(1-\alpha\).
\end{karte}

\begin{karte}{Elliptische Konfidenzbereiche}
Es gilt 
\[ \hat{\beta} \sim \mathcal{N}(\beta, \sigma^2 (X^T X)^{-1}) \]
\[\Rightarrow (\hat{\beta} - \beta)^T \frac{X^T X}{\sigma^2} (\hat{\beta}- \beta) \sim \chi_p^2 \text{ und } (n-p)\frac{\hat{\sigma}^2}{\sigma^2} \sim \chi_{n-p}^2 \text{ unabhängig.} \]

Somit folgt 
\[ \frac{(\hat{\beta} - \beta)^T (X^T X) (\hat{\beta} - \beta)}{p \hat{\sigma}^2} \sim F_{p,n-p}. \]
Der Bereich 
\[ B(y) := \set{ \tilde{\beta} \in \R^p: (\hat{\beta} - \tilde{\beta})^T (X^T X) (\hat{\beta} - \tilde{\beta}) \leq p \cdot \hat{\sigma}^2 \cdot F_{p,n-p;1-\alpha} } \]
erfüllt daher 
\[ P_{(\beta, \sigma^2)}(B(Y) \ni \beta) = 1-\alpha. \]
Da \(B(y)\) die Form eines Ellipsoids im \(\R^p\) hat, spricht man von einem \textit{elliptischen Konfidenzbereich} 
für \(\beta\) zur Wahrscheinlichkeit \(1-\alpha\).
\end{karte}

\subsection{Konf.intervalle Vorhersage}

\begin{karte}{Prädiktionswert}
Eine Vorhersage (sog. Prädiktionswert) für einen gegebenen Wert \(x_i\) erhält man durch 
\(\hat{y}_i = x_i^T \hat{\beta}\).

Eine Vorhersage für einen neuen Regressorvektor \(x_0\) durch \(\hat{y}_0 := x_0^T \beta\).
\end{karte}

\begin{karte}{Konfidenzintervall für \(x_0^T \beta\)}
Aus \(\hat{\beta} \sim \mathcal{N}(\beta, \sigma^2 (X^T X)^{-1})\) folgt für \(\hat{Y}_0 = x_0^T \hat{\beta}\)
\[ \hat{Y}_0 \sim \mathcal{N}(x_0^T \beta, \sigma^2 x_0^T (X^T X)^{-1} x_0). \]
Somit folgt mit \(\mathrm{se}(\hat{Y}_0) := \hat{\sigma} \sqrt{x_0^T (X^T X)^{-1} x_0}\)
\[ \frac{\hat{Y}_0 - x_0^T \beta}{\mathrm{se}(\hat{Y}_0)} \sim t_{n-p}. \]
Das Intervall 
\[ I := \left[ \hat{Y}_0 - \mathrm{se}(\hat{Y}_0) \cdot t_{n-p;1-\frac{\alpha}{2}}, \hat{Y}_0 + \mathrm{se}(\hat{Y}_0 \cdot t_{n-p;1-\frac{\alpha}{2}}) \right] \]
ist ein \((1-\alpha)\)-Konfidenzintervall für \(y_0 = x_0^T \beta\).
\end{karte}

\begin{karte}{Prädiktionsintervalle}
Wichtiger als Konfidenzintervalle für \(x_0^T \beta\) sind entsprechende Intervalle für den 
individuellen Vorhersagewert \(x_0^T \beta + \varepsilon_0\), \(\varepsilon_0 \sim \mathcal{N}(0, \sigma^2)\).

Analog zu den Konfidenzintervallen folgt aus 
\[ x_0^T \hat{\beta} - \varepsilon_0 \sim \mathcal{N}(x_0^T \beta, \sigma^2 x_0^T (X^T X)^{-1} x_0 + \sigma^2) \]
\[ \frac{x_0^T \hat{\beta} - (x_0^T \beta + \varepsilon_0)}{se_p} \sim t_{n-p}, \]
wobei \(se_p := \hat{\sigma} \sqrt{1+x_0^T (X^T X)^{-1} x_0}\) 
der zugehörige Standardfehler ist. 
Somit erhalten wir ein \((1-\alpha)\)-Prädiktionsintervall für \(x_0^T \beta + \varepsilon_0\) durch 
\[ \left[ \hat{Y}_0 - se_p \cdot t_{n-p;1-\frac{\alpha}{2}}, \hat{Y}_0 + se_p \cdot t_{n-p;1-\frac{\alpha}{2}} \right]. \]
\end{karte}